\documentclass[a4paper,12pt]{article}
\usepackage[english]{babel}
\usepackage[utf8]{inputenc}

%
% For alternative styles, see the biblatex manual:
% http://mirrors.ctan.org/macros/latex/contrib/biblatex/doc/biblatex.pdf
%
% The 'verbose' family of styles produces full citations in footnotes, 
% with and a variety of options for ibidem abbreviations.
%
\usepackage{csquotes}
\usepackage[style=verbose-ibid,backend=bibtex]{biblatex}
\bibliography{sample}

\usepackage{lipsum} % for dummy text

\title{Elaboration 1 and 2}

\author{John Hundley}

\date{\today}

\begin{document}
\maketitle

\section{Elaboration, Tools, and Revised Method:}

What will be principally tested is the feasibility of using a search engine to analyse multiple documents for key terms or themes. This was identified as a major problem in my research in my Scoping Exercise last week. Ideally, the aim is to perform a single search that might achieve this. The most appropriate software at this stage appears to be Voyant. Whilst this program appears not to perform a thematic analysis, it can produce word frequency analysis and frequency distribution graphs among a series of documents or databases. However, there might be a way to test a 'thematic' search in the broadest possible terms. If it cannot perform a basic thematic search then another program will have to be used instead. One benefit of this program is that it is an open-source project. It is also available on Git-Hub. The risks are that the difficulty of using the code of the program are unknown. If it is difficult to use then this presents several problems, namely available time and lack of skills. This is the first item that requires testing. Additionally, there are several other programs that also need to be tested such as Zotero and the relevant databases (Google Scholar and Macquarie University Library). The method as identified last week, has been revised to include the following: 


\begin{itemize} 

\item 1.	Identify thesis or argument. 
\item 2.	Identify relevant sources. 
\item 3.	Search necessary data bases. This includes Google Scholar and Macquarie University Library. 
\item 4.	Access and download required sources. This would require access to aforementioned databases. Sometimes this requires a university enrolment ID or a subscription service. 
\item 5.	Use Zotero toolbar when accessing files to record and manage bibliography.
\item 6.	Store sources in files kept on Couldstor and OneDrive. 
\item 7.	Open Voyant and upload sources in search engine. Perform search. 
\item 8.	Identify key themes, terms, and connections as demonstrated in Voyant. This might be indicated in the tables, graphs, and charts that Voyant provides. 
\item 9.	Collate this information and any metadata in a Word document, Zotero or Cloudstor file. Save work on Cloudstor. 
\item 10.	Read documents on either a PDF view or printed.
\item 11.	Write response on Word or Cloudstor. Work to be saved to computer hard-drive and Cloudstor folder. 
\item 12.	Use Zotero to generate bibliography. 

\section {Testing}

\textbf {Voyant:}

\item To test Voyant, I searched three documents relating to my research area for the term ‘Dionysian’. I wanted Voyant to provide the links between the documents of this term. Additionally, I wanted it to provide some further data about the term’s frequency, correlation, and distribution. So, I uploaded the three documents to Voyant's search engine. The engine then produced a new window where the three documents were analysed. I then selected 'Dionysian' from the 'Cirrus' section. There were a variety of other highly used terms here. Upon selecting ‘Dionysian’, Voyant informed me that it was used 118 times across the three papers. It also specified where the term was mentioned in the texts. Next, I went to the 'Terms' window. I selected 'Dionysian' from a list of other terms. From there I could see the relationship between that term and the three texts in the 'Trends' window across the screen. Additionally, it gave me information on word count, frequency, and distribution. Then, in the bottom right hand corner, I was able to sort the term by correlative and statistical significance. It also showed what words ‘Dionysian’ was commonly linked with. This was all useful information. In summary, Voyant appears be able to search multiple documents for a single term effectively. However, its layout and user-interface is slightly confusing. Also, it does not offer a textual analysis as hoped originally for. 

\textbf {Macquarie University Library System:} 

\item Macquarie University’s Library system required testing as it was outlined in the above Elaboration. It will be an essential tool in conducting me research. To assess the appropriateness of the system, I entered a search term relevant to my research area and selected the first three random results to see whether I would be granted access to them. This would require Macquarie’s library system to have subscriptions to various journals and repositories. Upon entering my search term, I selected the first three documents. All were accessible. The journals were subscribed to by Macquarie University. The first was 'ROAD: Directory of Open Access Scholarly Resources'. The second was 'Taylor and Francis Online'. The third was 'JSTOR Arts and Sciences'. This shows a high probability that Macquarie's system has access to most high quality journals. There were no issues accessing the journals. Macquarie University library system is clearly an appropriate tool to use when conducting research. 

\textbf {Zotero (Online):} 

\item I had been writing out my bibliography manually for years. As identified in the scope and sequence and above elaboration exercise, I wanted to have a program write my bibliography and references for me. So, I dowloaded Zotero and Zotero Connector. Then accessed an article by Vandenabeele (2003) on Macquarie University Library system. Clicked the 'Save to Zotero' option on newly installed toolbar. The document was then saved to my Zotero library. I opened Zotero and accessed the document and it had all the necessary bibliographic information. There was an option to have the bibliography created in any desire format. Success. Zotero seems to be an easy reference generating tool. I was considering using OneNote if it was too difficult to use but it proved simple.  

\textbf {Zotero (Offline):}

\item Due to the nature of my research area, many of the sources and texts I am using come from hard-copy (offline) sources. Some of these have been uploaded by various means onto my hard-drive or cloud service so I can access them via my computer. However, I still needed to test whether Zotero could prouduce bibliographic entries from sources that have not been imported from online or via the Zotero Connector. So, I tried to import a file from my hard-drive. It said the action was not supported because the file was in the wrong format. The file was in a PDF format. It says it only supports files in RIS, Bib(La)Tex, and MODS that can be imported. I then tried the 'Link to File' option on the same file. It appeared in Zotero, however, it was completely lacking any details other than the article name and author. This is a significant problem for any files that I upload from sources accessed offline. It appears that I might have to manually enter in the details. It seems the source has to be added to the Zotero collection when accessed online straight away otherwise it will not work. Unfortunately, this is very similar to simply writing out the bibliographic details manually. Perhaps OneNote might offer a solution.  

\section {Conclusions}

The tools I have selected so far appear to suit my research needs. The only significant problem I have encountered is the difficulty of uploading offline sources into Zotero's bibliography generator. The next significant step that needs to be taken is familiarisation with the tools I have selected. This involves primarily involves understanding the repository of Voyant on GitHub and getting to know how it works behind the scenes. Additionally, another bibliography generator might be sought to assist with the difficulty of uploading texts from offline sources. 



\end{document}